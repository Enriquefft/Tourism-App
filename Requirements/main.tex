\documentclass[a4paper,12pt,oneside]{article}

%% Packages

\usepackage[spanish]{babel}          % Language library
\selectlanguage{spanish}

\usepackage{float}                   % Positioning

\usepackage{xcolor}                  % Colors

\usepackage{biblatex}                % Bibliography
\addbibresource{referencias.bib}

\usepackage{graphicx}                % Graphics
\graphicspath{ {./images/}}                         % Set image path

\usepackage{hyperref}                % Hyperlinks
\hypersetup{
    urlcolor=blue,                   % Hyperlink color
    colorlinks=true                  % Enable color links
}

%% Custom commands
\usepackage{custom}

\author{
  \begin{tabular}{cc}
      \\
  Chambilla Mamani, Edgar Moises \\
  Cortijo Gonzales, Luis Enrique \\
  Pacheco Espino, Franco Matias \\
  Alvarado Vargas, Fabian Martin \\
  Chamochumbi Gutierrez, Alexandro Martin \\
  Quiroz Maquin, Nincol Abraham \\
  Flores Teniente, Enrique Francisco \\
  \end{tabular}
}

\title{Requerimientos del Proyecto}
\course{Ingenieria de Software}
\date{\today}

\begin{document}

\makeatletter
  \begin{titlepage}
    \let\footnotesize\small
    \let\footnoterule\relax
    \let \footnote \thanks
    \setcounter{footnote}{0}
    \null\vfil
    \begin{center}
      \setlength{\parskip}{0pt}
      \vspace{-1.5cm}
      {\large \textbf{UNIVERSIDAD DE INGENIERÍA Y TECNOLOGÍA}\par}
      \vfill
     
      \href{https://www.utec.edu.pe}{\includegraphics[height=2.5cm]{images/logo_utec.png}}\\
      \vspace{0.5cm}
      {\large \coursename \par}
      \vfil
      
      \vspace{0.5cm}
      {\Large \bf \@title \par}
      \vfill
      \smallskip
      \vspace{0.5cm}
      {\large \textbf{AUTHOR(S)}\par}
      {\large \@author \par}
      \vspace{0.5cm}

      \ifdefined\hasProfesor
        
        {\large \bf DOCENTE RESPONSABLE \par}
        \smallskip
        {\large \profesorname \par}
        \vfill

      \fi

      \vspace{1.5cm}
      {\large Lima - Perú \par}
      {\large \@date \par}
    \end{center}
    \par
    \vfil\null
  \end{titlepage}
  \makeatother


\tableofcontents
\newpage

%% ============================================================================

\section{Requesitos Funcionales}

\subsection{Must have}
  \begin{itemize}
    \item Sistema de login para empresas \\
      Autenticación y autorización para empresas en base a identificadores únicos
    \item Sistema de register para empresas \\
      Proceso de validación por medio de identificadores únicos
    \item Registro de eventos \\
      Las empresas deben poder ingresar eventos por medio de un formulario desde su cuenta personal en la app. Estos deberan incluir fotos e información de estos.
    \item Selección de eventos \\
      Los usuario deben ser capaces de seleccionar multiples eventos de su interes, esto se debe almacenar de manera persistente
    \item Sistema de gamificación \\
      Brindar puntos en la app a los usuarios en función de su asistencia a locales y consumo de ofertas en negocios locales.
    \item Visualización de ruta \\
      El usuario debera ser capaz de visualizar una ruta a los eventos u locales seleccionados

  \end{itemize}

\subsection{Should have}
  \begin{itemize}
    \item Sistema de login y register para usuarios \\
      exclusivamente por medio de la api de google authentication
    \item Ubicación del usuario en tiempo real \\
      Mostrar en el mapa de la aplicación la ubicacion del usuario en tiempo real.
    \item Visualización de ofertas \\
      Los usuario deben ser capaces de visualizar las diversas ofertas correspondientes a eventos que hallan seleccionado o no, junto con la infomación detallada y restricciones de la oferta.
  \end{itemize}

\subsection{Could have}
  \begin{itemize}
    \item Autenticación simple para empresas \\
      posibilidad de usar la api de google para facil ingreso.
  \end{itemize}

\subsection{Won't have}
  \begin{itemize}
    \item a \\
  \end{itemize}

\section{Requesitos No Funcionales}

\subsection{Must have}
  \begin{itemize}
    \item Obtencion de ruta óptima \\
      La ruta debera ser obtenida por medio de la api de google maps
    \item Categorización y labeling de eventos \\
      Los eventos, al ser registrados por las empresas, deberan ser categorizados y se agegara un grupo de labels. Esta funcionalidad hara uso de las fotos relacionadas al evento.
    \item Jerarquización de eventos \\
      Los eventos mostrados al usuario deberan ser organizados en función de sus labels y las preferencias del usuario
    \item Almacenamiento de eventos ingresados por las empresas
      Se deberan almacenar los eventos con una velocidad de I/O inferior a 3 segundos.
  \end{itemize}

\subsection{Should have}
  \begin{itemize}
    \item Obtencion de ruta con enfoque en turismo \\
      La ruta debera ofrecer alternativas interesantes para el turista en base a eventos o locales que podrían ser de su interes.
    \item Sistema de califiación de eventos \\
      Los eventos deberan poder ser calificados por los usuarios, esto funcionara con un sistema tipo Proof of Steak
    \item Jerarquización de eventos en base a su califiación \\
      Los eventos mostrados al usuario deberan tener una jerarquización adicional en función de su califiación
  \end{itemize}

\printbibliography[title={Bibliography}]

\end{document}
